\section{Punteros}

Los punteros en C++ son herramientas poderosas que permiten manipular directamente las direcciones de memoria. Esto facilita el manejo eficiente de datos, ya que podemos acceder y modificar valores sin necesidad de copiar datos innecesariamente.

\subsection{Declaración de Punteros}

Para declarar un puntero, se usa el operador \texttt{*} junto con el tipo de dato al que el puntero apuntará.

\begin{lstlisting}[language=C++]
int *puntero;
\end{lstlisting}

En este ejemplo, \texttt{puntero} es un puntero a un entero. Para asignar la dirección de memoria de una variable al puntero, usamos el operador \texttt{\&}.

\begin{lstlisting}[language=C++]
int variable = 5;
int *puntero = &variable;
\end{lstlisting}

\subsection{Acceso al Valor de un Puntero}

Para acceder al valor en la dirección de memoria a la que apunta un puntero, se utiliza el operador \texttt{*} (desreferenciación).

\begin{lstlisting}[language=C++]
int a = 10;
int* p = &a;
int b = *p; // b es 10, el valor de a
*p = 20; // a ahora es 20
\end{lstlisting}

\subsection{Punteros y Arreglos}

Los punteros pueden facilitar el acceso a los elementos de un arreglo.

\begin{lstlisting}[language=C++]
int arreglo[5] = {1, 2, 3, 4, 5};
int* p = arreglo; // p apunta al primer elemento del arreglo
int a = *p; // a es 1
int b = *(p + 1); // b es 2
\end{lstlisting}

\subsection{Punteros a Funciones}

Los punteros también pueden apuntar a funciones, permitiendo llamadas indirectas a funciones.

\begin{lstlisting}[language=C++]
int suma(int a, int b) {
    return a + b;
}

int (*puntero)(int, int) = suma;
int resultado = (*puntero)(2, 3); // resultado es 5
\end{lstlisting}

\subsection{Punteros a Estructuras}

Es posible apuntar a estructuras y acceder a sus miembros usando punteros.

\begin{lstlisting}[language=C++]
struct Persona {
    std::string nombre;
    int edad;
};

Persona p = {"Juan", 20};
Persona* puntero = &p;
std::string nombre = puntero->nombre; // nombre es "Juan"
\end{lstlisting}

\subsection{Punteros a Punteros}

Los punteros pueden apuntar a otros punteros, formando niveles adicionales de indirectividad.

\begin{lstlisting}[language=C++]
int a = 10;
int* p = &a;
int** pp = &p;
int b = **pp; // b es 10
\end{lstlisting}

\subsection{Punteros a Constantes}

Un puntero puede apuntar a una constante, lo que impide modificar el valor apuntado a través del puntero.

\begin{lstlisting}[language=C++]
const double PI = 3.14159;
const double* p = &PI;
double pi = *p; // pi es 3.14159
\end{lstlisting}

\subsection{Punteros a Funciones Miembro}

Los punteros pueden apuntar a funciones miembro de una clase, y se utilizan con una sintaxis especial.

\begin{lstlisting}[language=C++]
    class Persona {
    public:
        void saludar() {
            std::cout << "Hola" << std::endl;
        }
    };
    
    void (Persona::*puntero)() = &Persona::saludar;
    Persona p;
    (p.*puntero)(); // Imprime "Hola"
\end{lstlisting}