\section{Variables y Arreglos}

\subsection{Enteros}

Los tipos de datos enteros en C++ incluyen \texttt{int}, \texttt{short}, \texttt{long} y \texttt{long long}. Estos tipos se utilizan para almacenar números enteros, positivos y negativos.

\subsubsection{Declaración e Inicialización}

\begin{lstlisting}[language=C++]
int a = 10;
short b = 5;
long c = 1234567890;
long long d = 1234567890123456789LL;
\end{lstlisting}

\subsubsection{Operaciones Básicas}

Las operaciones básicas con enteros incluyen suma, resta, multiplicación, división y módulo.

\begin{lstlisting}[language=C++]
int x = 10, y = 3;
int sum = x + y;      // Suma
int diff = x - y;     // Resta
int prod = x * y;     // Multiplicacion
int quot = x / y;     // Division
int mod = x % y;      // Modulo
\end{lstlisting}

\subsection{Reales}

Los tipos de datos de punto flotante en C++ incluyen \texttt{float}, \texttt{double} y \texttt{long double}. Estos tipos se utilizan para almacenar números reales con decimales.

\subsubsection{Declaración e Inicialización}

\begin{lstlisting}[language=C++]
float pi = 3.14159f;
double e = 2.718281828459045;
long double bigNumber = 3.141592653589793238462643383279L;
\end{lstlisting}

\subsubsection{Operaciones Básicas}

Las operaciones básicas con números reales incluyen suma, resta, multiplicación, división y comparación.

\begin{lstlisting}[language=C++]
float a = 1.5f, b = 2.5f;
float sum = a + b;       // Suma
float diff = a - b;      // Resta
float prod = a * b;      // Multiplicacion
float quot = a / b;      // Division
bool isEqual = (a == b); // Comparacion
\end{lstlisting}

\subsection{Caracteres}

El tipo de datos \texttt{char} se utiliza para almacenar caracteres individuales. También se pueden utilizar para representar cadenas de caracteres (\texttt{std::string}).

\subsubsection{Declaración e Inicialización}

\begin{lstlisting}[language=C++]
char letter = 'A';
std::string greeting = "Hello, World!";
\end{lstlisting}

\subsubsection{Operaciones Básicas}

Las operaciones básicas con caracteres incluyen asignación, concatenación y comparación.

\begin{lstlisting}[language=C++]
char ch1 = 'A', ch2 = 'B';
bool isEqual = (ch1 == ch2); // Comparacion

std::string str1 = "Hello";
std::string str2 = "World";
std::string str3 = str1 + ", " + str2 + "!"; // Concatenacion
\end{lstlisting}

\subsection{Archivos}

La manipulación de archivos en C++ se realiza mediante las bibliotecas \texttt{<fstream>}, \texttt{<ifstream>} y \texttt{<ofstream>}. Estas permiten leer y escribir archivos.

\subsubsection{Lectura de Archivos}

\begin{lstlisting}[language=C++]
#include <fstream>
#include <iostream>
#include <string>

int main() {
    std::ifstream inFile("input.txt");
    std::string line;
    if (inFile.is_open()) {
        while (std::getline(inFile, line)) {
            std::cout << line << std::endl;
        }
        inFile.close();
    } else {
        std::cerr << "Unable to open file";
    }
    return 0;
}
\end{lstlisting}

\subsubsection{Escritura de Archivos}

\begin{lstlisting}[language=C++]
#include <fstream>
#include <iostream>

int main() {
    std::ofstream outFile("output.txt");
    if (outFile.is_open()) {
        outFile << "Hello, file!" << std::endl;
        outFile.close();
    } else {
        std::cerr << "Unable to open file";
    }
    return 0;
}
\end{lstlisting}

\subsection{Arreglos}

Los arreglos en C++ se utilizan para almacenar colecciones de elementos del mismo tipo. Pueden ser de tamaño fijo o dinámico (utilizando \texttt{std::vector}).

\subsubsection{Arreglos de Tamaño Fijo}

\begin{lstlisting}[language=C++]
int fixedArray[5] = {1, 2, 3, 4, 5};

// Acceso a elementos
int firstElement = fixedArray[0];
int thirdElement = fixedArray[2];
\end{lstlisting}

\subsubsection{Vectores (Arreglos Dinámicos)}

\begin{lstlisting}[language=C++]
#include <vector>
#include <iostream>

int main() {
    std::vector<int> dynamicArray = {1, 2, 3, 4, 5};
    dynamicArray.push_back(6); // Aniadir un elemento al final

    // Acceso a elementos
    int firstElement = dynamicArray[0];
    int thirdElement = dynamicArray[2];

    // Iterar sobre el vector
    for (int i : dynamicArray) {
        std::cout << i << " ";
    }
    std::cout << std::endl;

    return 0;
}
\end{lstlisting}

\subsection{Arreglos Multidimensionales}

\begin{lstlisting}[language=C++]
int matrix[3][3] = {
    {1, 2, 3},
    {4, 5, 6},
    {7, 8, 9}
};

// Acceso a elementos
int element = matrix[1][2]; // Elemento en la segunda fila, tercera columna
\end{lstlisting}